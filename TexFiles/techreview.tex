%!TEX root = project.tex

\chapter{Technology Review}
About seven to ten pages.

%!TEX root = project.tex

\section {Unity 3D}
Unity 3D is a a game development environment created by David Helgason who setup Unity Technologies. 59\% of all virtual reality games are made using the Unity 3D engine. Unity works on 30 platforms, including Windows, iOS, Android, Nintendo Switch, Playstation 4, Oculus Rift, etc. Unity competes with Epic’s Unreal Engine, the game engine behind Fortnite and many games on the PS4 and Xbox One. 50\% of all games are made using Unity 3D. 50\% of all games are made using Unity 3D. As we are making a virtual reality game, the two engines we could use are the Unity 3D engine or the Unreal 4 engine. Though the Unreal engine creates very beautiful games and is very well optimised, the programming language is C++. We simply haven't used C++ yet and starting a new language would make the process more complex. The Unity engine used the C# language, which we have used previously.

\section {Unity UNET}
UNet is a feature in Unity that allows the programmer to create multiplayer or networked games over a network. It accomplishes this by allowing one player or service act as a host or dedicated host. A standard host is also a client, a dedicated host is just a server and does not have a client connection. Each client sends data to the host and receives all data from the host instead of communicating with the other clients. UNet as of 2018 is deprecated and is planned to be replaced, although there is no information yet on its replacement. Part of the reason for UNet being replaced is due to data being sent from client to server, then to all other clients has a latency issue as expected due to the path the data uses, as latency is usually the amount of time it takes for data to travel from client to server and back. This can be seen even in our program, when a player jumps their character, the players character can be seen stuttering but doesn't have such a noticeable effect when the character is not jumping.

%!TEX root = project.tex

\section {Oculus Rift}
The Virtual Reality headset we used for our project is an Oculus Rift, this is one of the top head mounted displays on the market. Along with the Rift, Oculus has also released two development kits before releasing the Rift. They have released a mobile VR kit known as the Oculus Go. In spring 2019 Oculus will also release two more units, one known as the Rift S that will be similar to the Rift and one other unit known as the Oculus Quest, which will be a stand alone unit that will not require a computer to operate. \newline

The Rift connects to a gaming graphics card with a HDMI port and a USB 3.0 connection for the head mounted display. The headset comes with two touch controllers and two external sensors that also each require a USB 3.0 connection. Due to the problems that occur when using only two sensors, a third sensor is recommended for a much smoother experience. The position of the controllers and the headset in 3D space is captured using the external sensors.

\section {Java}
Java is a high-level programming language developed by Sun Microsystems. It was originally designed for developing programs for set-top boxes and handheld devices, but later became a popular choice for creating web applications. The syntax is similar to that of C++ but it is strictly an object orientated programming language. Java programs contain classes and methods. Classes are used to define objects and methods are assigned to individual classes.\newline

Java is also known for being more strict than C++, meaning variables and functions must be explicitly defined. This means Java source code may produce errors or "exceptions" more easily than other languages, but it also limits other types of errors that may be caused by undefined variables or unassigned types.\newline

Java programs are not run directly by the operating system. Instead, Java programs are interpreted by the JVM, Java Virtual Machine, which can be run on multiple platforms. This means all Java programs can be run on different platforms such as Mac, Windows and Unix computers.  In order for Java applications or applets to run at all the JVM must be installed.\newline

\section {Go Lang}
Go Lang was conceived by Robert Griesemer, Rob Pike and Ken Thompson when they started sketching the goals for a new language on a white board on September 21, 2007. By January 2008, Ken Thompson had started working on a compiler with which to explore ideas. The compiler generated C code as its output. By mid 2008 the language had become a full-time project and had settled enough to attempt a production compiler. Go became a public open source project on November 10, 2009. There are now millions of Go programmers, or gophers, around the world. Go's success has far exceeded all expectations.\newline

At the time of Go's inception, only a decade ago, the programming world was different from today. Production software was usually written in C++ or Java, GitHub did not exist, most computers were not yet multiprocessors, and other than Visual Studio and Eclipse there were few IDEs or other high-level tools available at all, let alone for free on the Internet. Throughout its design, Google have tried to reduce clutter and complexity. There are no forward declarations and no header files; everything is declared exactly once.\newline


\section {Python}
The Python programming language was created by dutch programmer Guido Van Rossum who started developing the new script in the late 1980s and finally introduced the first version of that programming language in 1991.\newline


Here are some of the principles that python is based on:\newline
\begin{itemize}
\item Beautiful is better than ugly
\item Simple is better than complex
\item Complex is better than complicated
\item Readability counts
\item In the face of ambiguity, refuse the temptation to guess
\item There should be one, and preferably only one, obvious way to do it
\item If the implementation is hard to explain, it’s a bad idea
\end{itemize}
Often people assume that the name Python was written after a snake. Even the logo of Python programming language depicts the picture of two snakes, blue and yellow.  But, the story behind the naming is somewhat different. Back in the 1970s, there was a popular BBC comedy tv show called Monty Python’s Fly Circus and Van Rossum happened to be the big fan of that show. So when Python was developed, Rossum named the project ‘Python’.\newline
\section {MongoDB}
Mongodb is an open source non relational database management system. MongoDB was created by Dwight Merriman and Eliot Horowitz, who had encountered development and scalability issues with traditional relational database approaches while building web applications at DoubleClick, an online advertising company that is now owned by Google Inc. The name of the database was derived from the word humongous to represent the idea of supporting large amounts of data.\newline

MongoDB uses collections and documents unlike relational databases that use tables and rows. A record in MongoDB is a document that contains a data structure made up of field and value pairs. MongoDB documents use a varient similar to JavaScript Object Notation (JSON) called Binary JSON known as BSON that allows for more data types. The fields in the documents are similar to columns in a relational databse and they can hold a large variety of data including other documents to arrays of documents.\newline

Documents in MongoDB must use a primary key as a unique identifier.
The (id field) is added by MongoDB to uniquely identify the document in the collection i.e.(id, Name, Address) id is the object id/primary key. Collections contain sets of documents and act as the equivalent of tables in a relational database.\newline

MongoDB does not require predefined schemas and it stores any type of data. Tihs provides scalability and felxibility compared to relational databases which makes it a useful database for companies running big data applications. One of the advantages of using documents is that these objects map to native data types in a number of programming languages. Also, having embedded documents reduces the need for database joins.\newline

MongoDB is available in community and commercial versions through vendor MongoDB Inc. MongoDB Community Edition is the open source release, while MongoDB Enterprise Server brings added security features.\newline
\cite{S174228761630031720160601}
\bibliographystyle{plain}
\section {Redis DB}
Redis (REmote DIctionary Server) is an open source (BSD licensed), in-memory data structure store, used as a database. It supports a large variety of different types of data that can be stored ranging from strings and lists to hashes and sorted sets.Redis allows you to run atomic operations on these types, like appending a string; increment the value in a hash; pushing an element to a list; computing set intersection, union and difference; or getting the member with highest ranking in a sorted set.
\newline

Key features of redis are:\newline
High-Level Data Structures:  Provides five possible data types for values: strings, lists, sets, hashes, and sorted sets. Operations that are unique to those data types are provided and come with well documented time-complexity (Big O notation).\newline

High Performance:  Due to its in-memory nature, the project maintainer’s commitment to keeping complexity at a minimum, and an event-based programming model, Redis boasts exceptional performance for read and write operations.\newline

Lightweight With No Dependencies:  Written in ANSI C, and has no external dependencies. Works well in all POSIX environments. Windows is not officially supported, but an experimental build is provided by Microsoft.\newline

High Availability:  Built-in support for asynchronous, non-blocking, master/slave replication to ensure high availability of data. There is currently a high-availability solution called Redis Sentinel that is currently usable, but is still considered to be a work in progress.\newline

\section {MariaDB}
MariaDB is an open source relational database management system (DBMS). MariaDB is the compatible drop-in replacement for the widely used MySQL database technology. MariaDB intends to have high compatibility with MySQL and exact matching with MySQL APIs and commands. MariaDB's API and protocol are compatible with those used by MySQL, and also some other features to support local non-blocking operations and progress reporting.\newline

MariaDB is developing continuously and any new updates are transmitted to end users very fast with updated features like bug tracking that can be viewed in detail. They also offer a cluster database intended for commercial use that enables multi-master replication.\newline

MARIADB VS MYSQL\newline
\begin{itemize}
\item MariaDB has superior query performance.
\item MariaDB has a more open source attitude.
\item Making the switch to MariaDB is very easy.
\item Galera implementation is better in MariaDB.
\item MariaDB is available as an option with some hosting environments, like RackSpace Cloud.
\end{itemize}
\newline
\section {Heidi}
\newline

\section {AWS}
Amazon Web Services (AWS), the pioneering cloud computing platform provided by Amazon.com, emerged from separate internal initiatives at Amazon over 16 years ago to both aid developers and also improve the efficiency of the company’s own infrastructure.\newline

Publicly launched on March 19, 2006, AWS offered Simple Storage Service (S3) and Elastic Compute Cloud (EC2). By 2009, S3 and EC2 were launched in Europe, the Elastic Block Store (EBS) was made public, and a powerful content delivery network (CDN), Amazon CloudFront, all became formal parts of AWS offering. These developer-friendly services attracted cloud-ready customers and set the table for formalized partnerships with data-hungry enterprises such as Dropbox, Netflix, and Reddit, all before 2010.\newline

\section {Docker}
Docker is an open platform for developing, shipping, and running applications. Docker enables you to separate your applications from your infrastructure so you can deliver software quickly. With Docker, you can manage your infrastructure in the same ways you manage your applications. Docker provides the ability to package and run an application in a loosely isolated environment called a container. The isolation and security allow you to run many containers simultaneously on a given host. Containers are lightweight because they don’t need the extra load of a hypervisor, but run directly within the host machine’s kernel. This means you can run more containers on a given hardware combination than if you were using virtual machines. You can even run Docker containers within host machines that are actually virtual machines.

\section {JSON}
\newline



