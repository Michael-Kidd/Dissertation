%!TEX root = project.tex

\newpage
\paragraph{Abstract}
We have created a cross platform multiplayer video game that can be played by using a mobile device, a traditional desktop or laptop and can be played using a virtual reality headset. The object of the project is to create a heterogeneous system that demonstrates an ability to communicate between multiple languages, operating systems and databases. This will be achieved through the use of data serialisation and marshalling and un-marshalling of data. The project will involve four programmers, using four different programming languages and 3 different databases. We also intended to deploy the system back-end within an AWS server and then deploy these separate systems within docker.

\paragraph{Authors}
The authors of this Document are Michael Kidd, John Mannion, Raymond Mannion and Kevin Moran, The authors are 4th year students in Galway Mayo Institute of Technology.

\paragraph{Github Link} \url{https://github.com/Michael-Kidd/4th-Year---Main-Project}

\chapter{Introduction}
The idea for our project came from a video that showed a version of Mario Kart in virtual reality. It depicted the fan favourite Mario Kart racing game that Nintendo originally released august 27 1992, but this time with a player sitting into a specially designed kart, placing a head mounted display unit on their heads and proceeding to play the game. There are currently 8 Mario Kart games, not including the VR versions or arcade versions. The unfortunate thing with this version of the game was that it was not open to the general public, it is simply an arcade version of the game and can only be played at designated venues.\newline

The project that we decided on was to create a similar style game to Mario Kart in virtual reality, however we also wanted that the game could be played cross-platform with a computer and also with mobile devices, while maintaining the ability for each of these versions to be able to play within the same instance. This would mean that a person playing the game on a mobile device and a person playing on a virtual reality headset could play together. From the very start we decided to make different versions of the game that would still allow the games to be played within the same instances. By separating the versions of the game we ensure that each version does not require nor depend on the packages and API's that the other versions depend on, for example the mobile and desktop versions would not depend on the Oculus API to function and therefore should not contain the files at all. \newline

There are also other kart games that already exist within virtual reality, however they have not been very popular as they often fall short in their implementation, such things as have no hand movement with the game. Forcing the player to use a steering wheel, while not being completely immersion breaking, is still not perfect. The games often are single player, removing the need to program for a multiplayer environment. Another issue with many virtual reality games is that the player base in virtual reality is simply too small, as the equipment needed to play is expensive, the setup of which takes up a large amount of space which can leave many people having to spend most of their time, assembling and dismantling their virtual reality rigs. Not to mention the expensive gaming PC that a player must already own to play the games, that are also sold separately. All in all, using virtual reality for gaming is expensive and the owner must have a desire to look past its limitations and requirements to want this. Another issue with virtual reality gaming is that players with virtual reality headset often can not play games with their friends who don't own the same technology. This is why when designing this game we decided that it would be important to have a cross platform game that didn't have any limitations on who could play with each other. For us, it was important that we show how a person moving their body and hands in virtual reality could be seen moving in real time.
\newline

We decided to create the main game within the Unity engine and using the C\# programming language, this will contain different scenes within the program that will serve different functions, the first being a login service. It will allow a user to create an account, then when a user has an account of their own it will allow the user to enter these details in order to login to the game. Since the game will be cross platform, a user can login to any of the different platforms and use the same credentials to gain access to the game. The login service would be programmed using the Java programming language and it will then use a MongoDB non relational database to be able to add new users when someone creates a new account. It will allow a user to access the game when they have entered the correct details. When the player enters incorrect details, they are refused entry to the game. After a successful login, the user will be presented with a screen where the user can host a game or find a game that is already being hosted by another user.\newline  

The match finding service will be programmed using the Go programming language and will connect to a Redis Database. The purpose of this system will be to keep a record of all games that are being hosted, it will keep a record of the username of the person hosting the game and that persons IP address. This will allow other players to see a list of hosted games and a list of player names but will not need the persons IP address in order to access the game. Once the user selects a game that they wish to join or host, they will be sent to a lobby screen, where they will be able to see the list of other players that are playing the same game. When all players have clicked a button to signal they are ready, the game will start.\newline 

The score keeping service will be programmed using the Python programming language and will connect to a MariaDB database. When the game has ended the players positions in the race will determine the score that they are given and their overall scores will be added to a database and sorted in order of how many points the players have accumulated. The player with the most points will be the first on the list and the player with the least points will be last in the list. When the players have completed the game they will see the position each player finished, then each players global position in the leader-board.\newline

With the back-end of the system for such features as user login, finding an active game and for global score keeping we intended to create 3 different programs that would be running within an Amazon web services server. Within the server we will run one instance of docker with 3 different containers. Each container will house one of the programs and its corresponding database, either the Java login service with the MongoDB instance, the Go match finding service with the Redis database, or the Python scoring service with the MariaDB database. \newline \newline

Part of the problems that will be faced when intentionally creating a heterogeneous system, is that there will be issues with data serialisation and with marshalling and un-marshalling of the data from one language to another. The C\# program will be communicating with 3 different programs that will act as services to accomplish specific tasks. The connections between the client side C\# program and each of the services will be made using an IP address and port number (socket) combination. Then once the connections are made the objects that will be added to the database or that may need to be manipulated by the service programs, will be passed through these sockets as serialised objects using either JSON or XML. The purpose of these formats is that they can be used by many different languages and frameworks as they are platform independent. This does not mean that it is guaranteed to work, along the way there is bound to be issues that we encounter and that we have yet to account for in our planning. As this is a college project, the principle of which is to develop us as programmers and to take us out of the areas that we have grown comfortable, we believe this project will present us with enough of a challenge to help us develop and also be within a realistic target so that we can have a product at the end that is at least a minimal viable product that works as a game. \newline

We don't expect that we will have a completely polished product at the end, and are aware of our own limitations as programmers and that we have not reached the level of experience that can be gained through working in the real world. That we don't yet understand all the possible bugs a real finished product would face or the vulnerabilities such a product would contain. Therefore we could not anticipate completely how our system could be interfered with or manipulated. In a worse case scenario, if it would be possible for all the players data to be intercepted. If such an attempt would be possible then the usernames, password and IP addresses of users could be compromised. During the programming phase of the project we simply made the game and its features function, we did not make a conscious attempt at protecting the data or transfers within the services or within the game itself. We are aware of the security issues and if the project was being released to the public, we would would have made security a priority. But, as the project is for educational purposes, we have decided to leave this feature out.



